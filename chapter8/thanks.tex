%# -*- coding:utf-8 -*-
\begin{thanks}

五年前,我从大连理工大学毕业,怀着对博士的无限憧憬,加入浙江大学计算机学院数字媒体计算与设计实验室(Digital media Computing \& Design Lab, DCD),成为一名直博生。时间飞逝,一眨眼我来杭城求学已经整整五个年头,回顾过去的五年,真是感慨万千。五年间,我也曾十分迷茫、痛苦甚至绝望。我曾无数次地想过中途放弃,但身边家人、朋友以及老师们对我始终如一的鼓励和支持,让我最终选择坚持了下来。这段难忘的求学时光,将成为我人生中最为宝贵的一笔精神财富。此时此刻,我想感激曾经帮助过我的每一个人。你们是我不断拼搏前进的动力。

首先,我最想感激的人是我的妻子,黄佳。为了支持我顺利完成学业,在我入学第一年就选择了辞职,放弃了自己的工作,重新选择攻读研究生。在上海工作期间,因为怕耽误我的科研时间,总是选择自己乘坐高铁来杭州看我。如今,为了女儿的健康成长,你又选择了全职在家照顾。没有你的爱,我无法走到现在。此份情意,今生无以为报!

其次,我要感谢我的导师肖俊教授。五年前,因为各种机缘巧合,我有幸成为了肖老师的第一个博士生。肖老师对待自己的学生们,都如同自己的孩子一般,不仅关注学生的学习进步,更关心学生的健康成长。在科研上,肖老师给了我无限的自由。从来没有给我施加任何的科研压力,让我可以凭借自己的兴趣选择研究课题,并且尽自己一切的能力帮助我联系和申请国外的交流合作机会。在生活上,肖老师给我了无限的关心。在我压力大时,时常陪我聊天。在大家很久没有户外活动时,带领实验室同学们一起去西湖边跑步。肖老师不仅传授知识,还教育我如何与人相处。短短几年,收获颇丰。衷心感谢肖老师的付出和指导,能够成为您的学生,是我的荣耀。

感谢南洋理工大学张含望(Hanwang Zhang)教授、哥伦比亚大学张世富(Shih-Fu Chang)教授和新加坡国立大学蔡达成(Tat-Seng Chua)教授。张含望教授是我科研的领路人,从读论文、到构想想法、到设计实验方案、到写论文、再到做报告,张老师对我进行全方位细致的指导。感谢张老师的付出和帮助,让我有幸能够顺利地度过博士阶段。张世富老师和蔡达成老师作为领域资深专家,他们严谨的治学态度和谦逊的品格,以身作则地告诉我一个优秀的学者应当具备的品质。通过与你们的交流和合作,极大地促进了我科研水平的提升,帮助我加深对问题的思考和对科研的敬畏。

感谢课题组的庄越挺老师、吴飞老师、孟黎瑾老师、李玺老师、汤斯亮老师和赵洲老师在我求学期间对我的真诚帮助和耐心指导。感谢实验室的冯银付、齐天、林昌隆、陈刘策、张瀚之、陈铭洲、周桓等师兄师姐,感谢徐得景、李泽、张宋扬、高旭扬、吉炜等同学,感谢叶钰楠、肖少宁、唐作其、李星辰、金韦克、张凤达、李一萌、王禹潼、孟令涛、严薪、钱旭峰、金松、黄成越、黄一峰、邵飞飞、俞鑫、马文博、蒋志宏、高凯锋、洪暖欣等师弟师妹,感谢所有在新加坡和纽约的小伙伴们,博士这几年和你们一起奋斗的时光,将成为我人生中最美好的一段回忆。

最后,感谢我的父母和家人对我的养育和照顾。感谢你们始终如一地在背后默默的支持我、帮助我,在我低落时鼓励我、安慰我。你们无私的奉献,激励着我一直向前行。

谨以此文献给所有关心我、帮助过我的人。

\begin{flushright}
	\begin{minipage}{12em}
	\begin{center}
		陈隆
		\\ 二零二零年夏\ 于求是园
	\end{center}
	\end{minipage}
\end{flushright}

\end{thanks}
