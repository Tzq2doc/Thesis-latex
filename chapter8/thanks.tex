%# -*- coding:utf-8 -*-
\begin{thanks}


时光飞逝,我来杭城求学已经整整五个年头。

我时常回想,博士这五年,我到底收获了什么。五年前,我也曾踌躇满志、指点江山、企图改变世界,五年后,我心如止水,但求过着平凡的一生。五年前,我也曾坚持锻炼、每天跑步五公里,五年后,我体态臃肿,满脸油腻。五年前,我也曾徒手推导麦克斯韦方程组、傅立叶变换,五年后,我只能敲着键盘响不停。五年前,我在凌工路2号憧憬着毕业后美好的新生活,五年后,我依旧在老和山下憧憬着毕业后美好的新生活。在这段孤独求索的旅程中,我开阔了自己的眼界,锻炼了专业的技能,丰富了人生的阅历,完成了身份的转变。这一切,都离不开老师们的敦敦教诲,家人们的温馨关怀和朋友们的亲切陪伴。对此,我向您们表示衷心的感谢和美好的祝愿。

首先,谨以最诚挚的敬意感谢我的导师——肖俊教授。这些年,我始终庆幸自己能够成为肖老师的博士生,加入动画组这个大家庭。肖老师不仅为我们提供了世界一流的科研环境、资源和条件,同时又给予我们足够自由的研究空间。肖老师从来不给自己的学生们施加任何科研压力,反而更像父亲一般,关心我们的学习收获和健康成长。在我无数次迷茫困惑时,肖老师总是耐心地对我进行开导;在我想要放弃时,肖老师总是鼓励我继续前行;在我生活困难时,肖老师总是义不容辞地提供力所能及的帮助。肖老师不仅传授我知识,更教会我许多与人相处的道理,对我终身受益。衷心感谢肖老师对我的付出和指导,能够成为您的学生,是我的荣幸。

其次,我要感谢新加坡南洋理工大学张含望(Hanwang Zhang)教授。张老师是我科研上的领路人,从如何找论文、读论文、到构思想法、到设计实验方案、到写论文、再到做报告,张老师对我进行了全方位和细致的指导。张老师对我无私的付出和帮助,让我能够顺利地达到学校的博士毕业要求。同时,张老师对待学术的认真态度,以身作则地展示一个优秀学者应当具备的品格,让我受益匪浅。

感谢美国哥伦比亚大学张世富(Shih-Fu Chang)教授和新加坡国立大学蔡达成(Tat-Seng Chua)教授。张老师和蔡老师作为领域内的资深专家,他们严谨的治学态度和谦虚的品格,为我树立了科研标杆。通过与您们的交流和合作,极大地促进了我科研水平的提升,帮助我加深对问题的思考和对科研的敬畏,对我今后的学术研究和工作生活带来了深远意义。

感谢母校大连理工大学的卢湖川教授和课题组的李玺教授,在您们的推荐下,让我五年前有机会加入数字媒体计算与设计实验室(Digital media Computing \& Design, DCD Lab)。感谢新加坡国立大学的尚辛迪博士,让我深切地明白自己离一个优秀的计算机博士之间的差距。感谢课题组的陈刘策师兄,在我刚进入实验室时对我各种问题的细致解答,帮助我顺利地适应研究生生活。

感谢课题组的庄越挺老师、吴飞老师、孟黎瑾老师、汤斯亮老师和赵洲老师在我求学期间对我的真诚帮助和耐心指导。感谢所有在杭州、新加坡和纽约的小伙伴们,博士这几年和你们一起奋斗的时光,将成为我人生中最美好的一段回忆。

感谢我的家人对我的养育和照顾。感谢你们始终如一的在背后默默支持我、帮助我,在我低落时鼓励我、安慰我。你们无私的奉献,激励着我一直向前行。

感谢所有关心我、帮助过我的人。

最后,我来回答一下竺可桢老校长的两个问题:

第一,到浙大来做什么?

树立正确的三观、找到人生的方向。

第二,将来毕业后做什么样的人?

对社会有益、生活幸福的人。

\textbf{\kaishu{谨以此文献给我的爱人:黄佳。}}

\begin{flushright}
	\begin{minipage}{12em}
	\begin{center}
		陈隆
		\\ 二零二零年夏\ 于求是园
	\end{center}
	\end{minipage}
\end{flushright}

\end{thanks}
