%# -*- coding:utf-8 -*-
\begin{publications}

% \section*{个人简历:}
% 姓名:陈隆 \qquad\qquad\qquad\qquad\qquad\qquad 出生年月:1993年12月

% 民族:汉族 \qquad\qquad\qquad\qquad\qquad\qquad 政治面貌:中共党员

% 邮箱:longc@zju.edu.cn \qquad\qquad\qquad\hspace{0.3em}  个人主页:\href{http://zjuchenlong.github.io}{zjuchenlong.github.io} 
 
% \vspace{1.0em}
% 教育经历:
% \begin{asparaitem}                                                
% \item{2015.09 – 2020.06: \quad 浙江大学 \qquad\qquad 计算机科学与技术学院 \qquad       直博}
% \item{2011.09 – 2015.06: \quad 大连理工大学 \qquad 信息与通信工程学院 \qquad\quad 本科}  
% \end{asparaitem}

\section*{发表论文:}

\begin{enumerate}
\item{
第一作者, IEEE Conference on Computer Vision and Pattern Recognition (CVPR), 2017.(CCF A 类)
}

\item{
第一作者. IEEE Conference on Computer Vision and Pattern Recognition (CVPR), 2018.(CCF A 类)
}

\item{
第一作者. IEEE International Conference on Computer Vision (ICCV), 2019.(CCF A 类)
}

\item{
第一作者. Thirty-Fourth AAAI Conference on Artificial Intelligence (AAAI), 2020.(CCF A 类)
}

\item{
第一作者. IEEE Conference on Computer Vision and Pattern Recognition (CVPR), 2020.(CCF A 类)
}

\item{
第二作者. Conference on Empirical Methods in Natural Language Processing (EMNLP), 2019.(CCF B 类)
}

\item{
第二作者. ACM International Conference on Multimedia (ACM MM), 2019.(CCF A 类)
}

\item{
第四作者. Neural Processing Letters, 2019. (SCI期刊). 
} 
\end{enumerate}



% \begin{enumerate}
% \item{
% %Long Chen, Hanwang Zhang, Jun Xiao, Wei Liu, Shih-Fu Chang.
% Zero-Shot Visual Recognition using Semantics-Preserving Adversarial Embedding Networks[C].
% IEEE Conference on Computer Vision and Pattern Recognition (CVPR), 2018.
% % In CVPR, 2018.
% (第一作者, CCF A 类)
% }

% \item{
% %Long Chen, Hanwang Zhang, Jun Xiao, Xiangnan He, Shiliang Pu, Shih-Fu Chang.
% Counterfactual Critic Multi-Agent Training for Scene Graph Generation[C].
% IEEE International Conference on Computer Vision (ICCV), 2019.
% % In ICCV, 2019.
% (第一作者, CCF A 类)
% }

% \item{
% %Long Chen, Hanwang Zhang, Jun Xiao, Liqiang Nie, Jian Shao, Wei Liu, Tat-Seng Chua.
% SCA-CNN: Spatial and Channel-wise Attention in Convolutional Networks for Image Captioning[C].
% IEEE Conference on Computer Vision and Pattern Recognition (CVPR), 2017.
% % In CVPR, 2017.
% (第一作者,CCF A 类)
% }

% \item{
% %Long Chen, Chujie Lu, Siliang Tang, Jun Xiao, Dong Zhang, Chilie Tan, Xiaolin Li.
% Rethinking the Bottom-Up Framework for Query-based Video Localization[C].
% Thirty-Fourth AAAI Conference on Artificial Intelligence (AAAI), 2020.
% % In AAAI, 2020.
% (第一作者,CCF A 类)
% }

% \item{
% %Long Chen, Xin Yan, Jun Xiao, Hanwang Zhang, Shiliang Pu, Yueting Zhuang.
% Counterfactual Samples Synthesizing for Robust Visual Question Answering[C].
% IEEE Conference on Computer Vision and Pattern Recognition (CVPR), 2020.
% % In CVPR, 2020.
% (第一作者,CCF A 类)
% }

% \item{
% %Chujie Lu,Long Chen, Chilie Tan, Xiaolin Li, Jun Xiao.
% DEBUG: A Dense Bottom-Up Grounding Approach for Natural Language Video Localization[C].
% In Conference on Empirical Methods in Natural Language Processing (EMNLP), 2019.
% % In EMNLP, 2019.
% (第二作者, CCF B 类)
% }

% \item{
% %Lei Meng,Long Chen, Xun Yang, Dacheng Tao, Hanwang Zhang, Chunyan Miao, Tat-Seng Chua.
% Learning Using Privileged Information for Food Recognition[C].
% In ACM International Conference on Multimedia (ACM MM), 2019.
% % In ACM MM, 2019.
% (第二作者,CCF A 类)
% }

% \item{
% % Shaoning Xiao, Yimeng Li, Yunan Ye,Long Chen, Shiliang Pu, Zhou Zhao, Jian Shao, Jun Xiao.
% Hierarchical Temporal Fusion of Multi-grained Attention Features for Video Question Answering[J]. 
% Neural Processing Letters, 2019. 
% (第四作者,SCI期刊)
% } 
% \end{enumerate}

% \section*{参与项目:}
% \begin{enumerate}
% \item{基于自适应特征学习和表观建模的目标跟踪算法研究,国家自然科学基金面上项目,2015/01 - 2018/12,61472353}
% \item{面向公共安全的跨媒体计算理论与方法,国家重点基础研究发展计划(973计划),2012/01 - 2016/12,2012CB316400}
% \item{城市智慧安监的相关基础理论和视觉分析技术,国家自然科学基金-浙江两化融合联合基金,2016/01 - 2019/12,U1509206}
% \item{中国工程科技知识中心关键技术研发,中国工程院工程科技知识中心建设项目,2013/01 - 2017/12,124001-D01703}
% \item{环境/场景适应的跨媒体综合推理,国家自然科学基金人工智能基础研究应急管理项目,2018/01 - 2020/12,61751209}
% \item{三元空间群智计算,国家重点基础研究发展计划(973计划),2014/11 - 2019/11,2015CB352302}
% \end{enumerate}


\section*{所获荣誉:}

\begin{itemize}
    
\item 2016-2017学年:优秀研究生、三好研究生

\item 2017-2018学年:优秀研究生、浙江大学博士研究生学术新星

\item 2018-2019学年:优秀研究生、三好研究生

\end{itemize}

\end{publications}
