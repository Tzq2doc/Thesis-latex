\chapter{基于反事实多智能体学习的图像场景图生成方法}

\section{问题描述}

\section{反事实多智能体学习}

\section{实验设置与性能对比}
\subsection{图像场景图生成数据集与实验设定}

\noindent{\kaishu{图像场景图生成数据集}}:我们使用目前最大的场景图数据集Visual Genome(VG) ~\cite{krishna2017visual}。为了与现有工作能够公平地进行比较,我们采用与现有工作相同的数据集划分和预处理~\cite{xu2017scene, zellers2018neural, newell2017pixels, yang2018graph, herzig2018mapping}。处理后的图像数据共包含150个物体类别和50个视觉关系类别。每张图像平均有11.5个物体和6.2个视觉关系。整个数据集中,70\%的图像数据当成训练集,30\%的图像数据当成测试集。


\noindent{\kaishu{实验设定}}:参考现有的文献~\cite{xu2017scene, zellers2018neural, jae2018tensorize},我们在三种实验设定下评估场景图生成质量:
\begin{asparaenum}
\item 视觉关系分类(PredCls):给定图像、所有的物体框和物体类别,模型需要预测所有的物体组合的视觉关系;

\item 场景图分类(SGCls):给定图像和所有的物体框,模型需要预测所有物体类别以及所有物体组合的视觉关系;

\item 场景图生成(SGDet):给定图像,模型需要检测物体框、预测所有物体类别以及所有物体组合的视觉关系。
\end{asparaenum}

对于视觉关系中物体框检测来说,需要主语(subject)和宾语(object)与真实准确物体框的交并比(IoU)均大于0.5。按照惯例,我们使用Recall@20(R@20)、Recall(R@50)和Recall(R@100)作为场景图生成质量的评价指标。


\subsection{实验细节}


\noindent{\kaishu{物体检测器}}:为了公平地与现有工作进行对比,我们采用了与~\cite{zellers2018neural}一样的物体检测器。具体来说,它是以VGG网络~\cite{simonyan2015very}为主干网络



\noindent{\kaishu{速度与正确率的权衡}}


\noindent{\kaishu{SGDet的后处理}}

\section{本章小结}